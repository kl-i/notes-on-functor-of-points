\documentclass[../main.tex]{subfiles}
\begin{document}

\begin{rmk}
  A scheme will be a 
  ``space modelled on $\M$ with an open cover by affine schemes''.
  This section defines the notion of ``open subfunctors'' 
  of a $\Z$-functor and an ``open cover''.
  Since $\Z$-functors are meant to be 
  modelled on $\M$,
  we first define everything for affine schemes.

  Somehow, no one could eradicate the 
  special role of fields as ``points''. 
  Perhaps it should be local rings instead.
  Something about points of the topos $\SH(\AFF_\zar)$
  corresponding to local rings\dots
\end{rmk}

\begin{dfn}[Points and Covers]\footnote{
  $\pts$ and ``surjective on points'' is non-standard definition.
  }
  % Let $X \in \M\SET$.
  % A \emph{point of $X$} is a morphism 
  % $x : \SP K \to X$ where $K$ is a field. 
  % For $x$ a point of $X$ and $\ph \in \M\SET(X,Y)$,
  % we write $\ph(x) := \ph \circ x$.
  Define the \emph{category of test points}, $\pts$, 
  to be the full subcategory of $\M$ consisting of $K\op$ where $K$ is a field.
  For $X \in \M\SET$, 
  a \emph{point of $X$} is defined to be 
  a morphism $x \in \M\SET(\SP K,X)$ where $K\op \in \pts$.
  For $\ph \in \M\SET(X,Y)$,
  we will use $\ph(x)$ to denote $\ph \circ x$.
  For $f \in \OO(X)$, 
  we will use $\ev_x$ to denote the pullback $x^\flat : \OO(X) \to \OO(K)$.

  % Let $\UU \subs \SUB\M\SET(X)$.
  % Then TFAE : 
  % \begin{enumerate}
  %   \item For all points $x : \SP K \to X$ of $X$,
  %   there exists $U \in \UU$ with a factoring 
  %   \begin{cd}
  %     \SP K \ar[r,"x"] \ar[rd] & X \\
  %     & U \ar[u]
  %   \end{cd}
  %   \item For all fields $K$, 
  %   $X(K) = \bigcup_{U \in \UU} U(K)$.
  % \end{enumerate}
  % We say $\UU$ \emph{covers} $X$ when it satisfies any 
  % (and thus all) of the above. 
  For $\ph \in \M\SET(Y,X)$,
  we will use $Y^\pts,X^\pts$ to denote the restriction of $Y,X$ to $\pts\op$
  and $\ph^\pts : Y^\pts \to X^\pts$ the restricted morphism.
  Then we say $\ph$ is \emph{surjective on points} when 
  $\ph^\pts$ is an epimorphism (equivalently, component-wise surjective).
  
  For a subset $\UU \subs \M\SET\darrow X$ for any $X \in \M\SET$,
  we say $\UU$ \emph{covers} $X$ when 
  $\coprod \UU \to X$ is surjective on points.
  Equivalently, for all points $x : \SP K \to X$ of $X$, 
  there exists $U \in \UU$ with a factoring 
  \begin{cd}
    U \ar[r] & X \\
    & \SP K \ar[u,"x"{swap}] \ar[ul]
  \end{cd}
\end{dfn}


\begin{prop}[Base Change of Cover]
  
  Let $\UU$ be a cover of $X \in \M\SET$.
  Then for all $\ph \in \M\SET(Y,X)$,
  the set $\ph\inv\UU$ of pullbacks of morphisms in $\UU$ 
  forms a cover of $Y$.
\end{prop}
\begin{proof}
  % Pullbacks of mono are mono so 
  % $\ph\inv\UU \subs \SUB\M\SET(Y)$.
  % Then ``preimage of union is union''.
  % Then for any point $x : \SP K \to Y$ of $Y$,
  % $\ph(x) : \SP K \to X$ is an point of $X$ and hence 
  % there exists $U \in \UU$ that $\ph(x)$ factors through. 
  % It follows from the UP of fiber products that 
  % $x : \SP K \to Y$ factors through $\ph\inv U$.
  Follows from fiber product in $\M\SET$ being computed component-wise.
\end{proof}

\begin{prop}[Multiplicative Group Scheme]
  
  Consider the functor $\G^\times \in \M\SET$ defined by 
  $A \in \M\op \mapsto A^\times$.
  Then 
  \begin{itemize}
    \item $\G^\times$ is representable by 
    the ring $\Z[T,T\inv]$ and hence affine.
    \item $\G^\times$ is a group object in $\M\SET$.
    In fact, for any $X \in \M\SET$,
    $\M\SET(X,\G^\times) = \OO(X)^\times$. 
  \end{itemize}
\end{prop}
\begin{proof}
  UP of $\Z[T,T\inv]$ implies it represents $\G^\times$.
  The second property can be straightforwardly deduced either from 
  the $\spec$-global functions adjunction or elementarily.
\end{proof}

\begin{prop}[Basic Opens of a $\Z$-Functor]
  
  Let $X \in \M\SET$ and $f \in \OO(X)$.
  The \emph{support of $f$}, $X_f$, is defined as 
  the subfunctor of $X$ sending 
  $A \in \M\op$ to the set of $\al \in \M\SET(\SP A, X)$ such that 
  $\ph^\flat(f) \in A^\times$.

  Then $X_f$ is the pullback of $\G^\times$ along $f : X \to \A^1$.
  \begin{cd}
    X_f \ar[r] \ar[d] & X \ar[d] \\
    \G^\times \ar[r] & \A^1
  \end{cd}

  Subfunctors of $X$ of the form $X_f$ are called \emph{basic opens}.
\end{prop}
\begin{proof}
  Easy.
\end{proof}

\begin{rmk}[Intuition of Multiplicative Group Scheme and Basic Opens]
  
  For a smooth manifold $X$,
  \[
    C^\infty\MFD(X,\R^\times) \iso C^\infty(X)^\times
  \]
  ``$\R^\times$ is classifying space for invertible global functions on $X$.''
  One can thus think of $\G^\times$ as ``$\A^1\minus\set{0}$''.
  A basic open $X_f$ is then just the preimage of ``$\A^1\minus\set{0}$''
  under $f : X \to \A^1$.
\end{rmk}

\begin{prop}[Opens of Affine Schemes]
  
  Let $X \in \AFF$.
  For $I \in \ideal \OO(X)$, 
  define $D(I) \in \SUB\M\SET(X)$ by 
  \[
    A \in \M\op \mapsto \set{\ph \in \AFF(\SP A, X) \st 
    A\ph^\flat I = A}
  \] 
  In particular, for $I = (f)$,
  we have $D(I) = X_f$.
  Sometimes, we use $D(f)$ to denote $X_f$.

  Then 
  \begin{itemize}
    \item (Basic Opens are Affine) 
    for $f \in \OO(X)$, $D(f)$ is representable by $\OO(X)[f\inv]$.
    \item (Ideals to Opens)
    For $I, J \in \ideal \OO(X)$,
    $I \subs J$ implies $D(I) \subs D(J)$.

    This defines $D : \ideal \OO(X) \to \SUB\M\SET(X)$.
    The category $\open X$ of \emph{opens of $X$} is defined as the 
    essential image of $D$.
    We call $U \in \open X$ an \emph{open of $X$}.
  
    \item (Intuitive Definition of Opens) for $I \in \ideal \OO(X)$,
    $\set{D(f)}_{f \in I}$ covers $D(I)$.
    \item (Partition of Unity)
    For $I \in \ideal \OO(X)$,
    $D(I)$ covers $X$ if and only if 
    there exists finite $I_0 \subs I$ such that $AI_0 = A$.
    Such $I_0 \subs A$ are called \emph{partitions of unity}.
    \item (Base Change / ``Preimage of Opens are Open'')
    Let $\ph \in \AFF(Y,X)$, $I \in \ideal \OO(X)$.
    Let the following be a pullback diagram : 
    \begin{cd}
      Y \ar[r,"\ph"] & X \\
      \ph\inv D(I) \ar[u] \ar[r] & D(I) \ar[u]
    \end{cd}
    Then $\ph\inv D(I) = D(\ph^\flat I)$.
    \item (Fiber Product)
    Let $X,X_1 \in \AFF/Y$ where $Y \in \AFF$.
    Let $I,I_1$ be ideals of $\OO(X)$, $\OO(X_1)$ respectively. 
    Then we have the pullback diagram : 
    \begin{cd}
      D(I) \ar[r] & Y \\
      D(I \otimes_{\OO(Y)} I_1) \ar[u] \ar[r] & D(I_1) \ar[u]
    \end{cd} 
    In particular, $X = X_1 = Y$ implies 
    the intersection of two opens of $X$ is an open of $X$.
    The special case of $D(I) = D(f)$ and $D(I_1) = D(f_1)$ yields 
    $D(f) \cap D(f_1) = D(f f_1)$.
  \end{itemize}

  
\end{prop}
\begin{proof}~

  \textit{(Basic Opens are Affine)}
  UP of $\OO(X)[f\inv]$ as an $\OO(X)$ algebra. 

  \textit{(Intuitive Def of Opens)}
  Let $x : \SP K \to X$ be a point of $X$.
  Then $Kx^\flat I = K$ if and only if there exists $f \in I$ with 
  $f(x) \in K^\times$.

  \textit{(Partition of Unity)}
  Having finite $I_0 \subs I$ with $AI_0 = A$ is equivalent to 
  $AI = A$.
  Clearly, $AI = A$ implies $D(I)$ covers $\SP A$.
  Conversely, suppose $AI \ssubs A$.
  $D(I)$ not covering $\SP A$ is the same as 
  it ``missing a point of $\SP A$'',
  that is to say we are looking for a point $x : \SP K \to \SP A$ of $\SP A$
  that doesn't admit a lift across $D(I) \to \SP A$.
  This is the same as $I \subs \ker\ev_x$.
  Well, $AI \ssubs A$ implies by Zorn's lemma the existence of 
  a map $\ev_x : A \to K$ where $K$ is a field with the desired property.
\end{proof}

\begin{ceg}[$\bigcup_{f \in I} D(f) = D(I)$]

  Consider the ring $\F_2 \times \F_2$ and elements $(1,0),(0,1)$.
  The ideal $I$ generated by these is the whole ring. 
  But 
  $D((1,0))(\F_2\times\F_2) \cup D((0,1))(\F_2\times\F_2) 
  \ssubs D(I)(\F_2\times\F_2)$ since 
  the ring endomorphism $(a,b) \mapsto (b,a)$ doesn't map 
  any of $(1,0),(0,1)$ to units. 
  Thus $D((1,0)) \cup D((0,1)) \ssubs D(I)$.
\end{ceg}

\begin{dfn}[Open Subfunctor]
  
  Let $X \in \M\SET$.
  For $U \in \SUB\M\SET(X)$,
  $U$ is called \emph{open} when 
  for all $\ph : \SP A \to X$, the pullback $\ph\inv U$ of $U$ along $\ph$
  is an open of $\SP A$.
  \begin{cd}
    \SP A \ar[r,"\ph"] & X \\
    \ph\inv U \ar[u] \ar[r] & U \ar[u]
  \end{cd}
  We will use $\open X$ to denote the 
  full subcategory of opens of $X$ in $\SUB\M\SET(X)$.
  When $X$ is affine, 
  this agrees with our specialized notion for affine schemes.
\end{dfn}

\begin{prop}[Basic Facts about Open Subfunctors]
  
  The following are true : \begin{itemize}
    \item (``Extensionality'')
    Let $U, V \in \open X$.
    Then $U = V$ if and only if $U^\pts = V^\pts$.
    \item (Composition)
    Let $V \in \open U$, $U \in \open X$, $X \in \M\SET$.
    Then $V \in \open X$.
    \item (Base Change/``Preimage of Opens are Opens'') 
    Let $X \in \M\SET$, $U \in \open X$ and $\ph \in \M\SET(Y,X)$.
    Then the pullback $\ph\inv U$ of $U$ along $\ph$ is 
    an open of $Y$.
    \begin{cd}
      Y \ar[r,"\ph"] & X \\
      \ph\inv U \ar[u] \ar[r] & U \ar[u]
    \end{cd}
    \item (Fiber Product)
    Let $U,U_1$ be opens of $X,X_1 \in \M\SET$ respectively.
    Then for any $X \to S, X_1 \to S$, 
    the fiber product $U \times_S U_1$ is an open of $X \times_S X_1$.
    In particular, for $X = X_1 = S$,
    this proves the intersection of two opens of $X$ is 
    again an open of $X$.
  \end{itemize}
\end{prop}
\begin{proof}
  (Extensionality) Reduce to affine global case and use partition of unity.
\end{proof}

\end{document}