\documentclass[../main.tex]{subfiles}
\begin{document}
  
\begin{dfn}[$\Z$-Functor]

  Define $\M := \Z\ALG\op$.
  The \emph{category of $\Z$-functors}
  $\M\SET$ is define to be the category of 
  presheaves of sets on $\M$. 
  For $X \in \M\SET$ and $A\op \in \M$,
  elements of $X(A\op)$ are called 
  \emph{morphisms from $A\op$ to $X$}.

  Define $\SP : \M \to \M\SET$ to be 
  the fully faithful Yoneda embedding $A\op \mapsto \M(-,A\op)$.
  For $A \in \Z\ALG$, $\SP A$ is called the \emph{spectrum of $A$}.
  The \emph{category of affine schemes} is defined to be 
  the essential image of $\SP$.
  We will denote it with $\AFF$.
\end{dfn}

\begin{rmk}[Intuition of $\Z$-Functors and Spectrums]
  In differential geometry, 
  given a morphism of smooth manifolds $\ph : X \to Y$,
  one obtains a ring morphism $\ph^\flat : C^\infty(Y) \to C^\infty(X)$
  that is the pullback of global smooth functions.
  For algebraic geometry, here's how to think about it : 
  \begin{itemize}
    \item $\M$ is the category of ``test spaces''.
    For smooth manifolds, this is open subsets of $\R^n$.
    \item A ring $A$ is the ``ring of global functions on $A\op$''.
    \item A ring morphism $\ph : A \to B$ is the ``pullback of global functions 
    along $\ph\op : B\op \to A\op$''.
    \item An $\Z$-functor $X \in \M\SET$ is something
    our ``test spaces'' in $\M$ can ``map into''.
    Essentially,
    \[
      \text{``$X(A\op) = \mor(A\op ,X)$''}
    \]
    Then given $\ph\op : B\op \to A\op$,
    one should be able to turn ``maps'' $\al : A\op \to X$ into 
    $\al \circ \ph\op : B\op \to X$.
    This is precisely $X(A\op) \to X(B\op)$ and 
    functoriality simply expresses how pre-composition respects 
    the identity morphisms and composition.
    \item The spectrum functor $\SP$ realizes 
    a ``space'' $A\op$ in $\M$ as something 
    our ``test spaces'' in $\M$ can map to.
    Then the intuition of ``$X(A\op) = \mor(A\op ,X)$''
    is realised as $X(A\op) \iso \M\SET(\SP A,X)$.
    This is Yoneda's lemma.
    \item $\AFF$ formalizes what we mean by ``test spaces in $\M$''.
    \item $\M\SET$ is complete and cocomplete as a category (since $\SET$ is),
    i.e. it's the perfect playground for building more general ``spaces''
    out of affine schemes.
  \end{itemize}
\end{rmk}

\begin{prop}[Yoneda]
  
  The following are true : 
  \begin{itemize}
    \item (``Morphisms from $A\op$ to $X$ $\longleftrightarrow$ 
    Morphisms from $\SP A$ to $X$'') 
    For $X \in \M\SET$ and $A\op \in \M$,
    \[
      \M\SET(\SP A, X) \map{\sim}{} X(A\op)
    \]
    by $\al \mapsto \al_{A\op}(\id{A\op})$.
    Furthermore, this is functorial in $A\op$ and $X$.
    \item (Density of Representables 
    / ``The data of $X$ is precisely how test spaces map into it'') 
    For $X \in \M\SET$,
    $X$ is the colimit of the diagram $\SP \darrow X \to \M\SET$.
  \end{itemize}
  
\end{prop}
\begin{proof}
  Straightforward.
\end{proof}

\begin{prop}[Affine Line]
  
  Let $n \in \N$.
  Define \emph{affine $n$-space} to be $\A^n \in \M\SET$
  sending $A\op \mapsto A^n$.
  Then
  \begin{itemize}
    \item $\A^n$ is representable by $\Z[T_1,\dots,T_n]\op$.
    Hence $\A^n \in \AFF$.
    \item for $n = 1$, $\A^1$ is a ring object in $\M\SET$.
    Hence for $X \in \M\SET$,
    $\OO(X) := \M\SET(X,\A^1) \in \Z\ALG$. 
    This is called the \emph{ring of global functions on $X$} and gives 
    a functor $\OO(\star) : \M\SET \to \Z\ALG\op$.
    We call elements of $\OO(X)$ \emph{functions on $X$}.

    Concretely on morphisms $\ph \in \M\SET(Y,X)$,
    the corresponding ring morphism $\ph^\flat : \OO(X) \to \OO(Y)$
    is given by $f \mapsto f \circ \ph$, i.e. 
    pulling back along $\ph$.
    
    \item (Spec, Global Function Adjunction)
    \[
      \M\SET(-,\SP \star) \iso \Z\ALG(\star,\OO(-))
    \]
    \item (Canonical Choice of Spectrum for Affine Schemes) for $X \in \M$, 
    $X$ is affine if and only if $\id{\OO(X)}^\bot : X \to \SP \OO(X)$ 
    is an isomorphism of $\Z$-functors,
    where $\id{\OO(X)}^\bot$ is the adjunct of $\id{\OO(X)}$ under the 
    previous adjunction.
  \end{itemize}
\end{prop}
\begin{proof}
  \textit{(Spec, Global Function Adjunction)}
  Follows from this chain of bijections functorial in $A$ and $X$ 
  given by the density of representables :
  \begin{align*}
    \M\SET(X,\SP A) 
    &\iso  \LIM_{(B,\be) \in \SP\darrow X} \M\SET(\SP B,\SP A) 
    \iso \LIM_{(B,\be) \in \SP\darrow X} \Z\ALG(A,B) 
    \iso \Z\ALG\brkt{A,\LIM_{(B,\be) \in \SP\darrow X} B} \\
    &\iso \Z\ALG\brkt{A,\LIM_{(B,\be) \in \SP\darrow X} \M\SET(\SP B,\A^1)} 
    \iso \Z\ALG(A,\M\SET(X,\A^1)) = \Z\ALG(A,\OO(X))
  \end{align*}

  \textit{(Affine Schemes)}
  The reverse implication is clear.
  Let $X \map{\sim}{} \SP A$ be an isomorphism in $\M\SET$.
  The adjunction $\M\SET(X,\SP A) \iso \Z\ALG(A,\OO(X)) 
  \iso \M\SET(\SP \OO(X),\SP A)$ gives the commutative diagram : 
  \begin{cd}
    X \ar[r,"\sim"] \ar[d] & \SP A \\
    \SP \OO(X) \ar[ru,dashed]
  \end{cd}
  where the dashed has the composition of the other two morphisms 
  as an inverse, and hence an isomorphism.
  
\end{proof}

\begin{rmk}[Intuition of Affine $n$-Space]
  
  For a smooth manifold $X$, 
  a smooth map $\ph : X \to \R^n$ is equivalent to the 
  data of $n$ global smooth functions $f_1,\dots,f_n$ on $X$, 
  i.e. \[
    C^\infty\MFD(X,\R^n) \iso C^\infty(X)^n
  \]
  ``$\R^n$ is the classifying space of $n$-tuples of global smooth functions.''
  In the functorial POV of algebraic geometry, 
  we take this as our definition of affine $n$-space.

\end{rmk}

\begin{prop}[Categorical Properties of $\Z$-Functors]
  
  The following are true : 
  \begin{itemize}
    \item (Completeness and Cocompleteness)
    $\M\SET$ has all (small) limits and colimits,
    all computed pointwise. 

    In particular, a morphism $\ph \in \M\SET(Y,X)$ is 
    respectively a mono/epi/isomorphism if and only if 
    for all $A \in \M\op$, $\ph_A : Y(A) \to X(A)$ is 
    injective/surjective/bijective.
    \item (Base Change)
    For $K \in \CRING$,
    define the \emph{category of $K$-functors} to be 
    the over-category $\M\SET/\SP K$.
    In particular, we call $\AFF/\SP K$ the 
    \emph{category of affine $K$-schemes}.

    Then we have for $\ph \in \M\SET(\SP L,\SP K)$,
    we have the following adjunction \[
      \brkt{\M\SET/\SP L} (-,\SP L \times_{\SP K} (\star)) \iso 
      \brkt{\M\SET/\SP K} (-,\star)
    \]
    Furthermore, this restricts to an adjunction between 
    $\AFF/\SP L$ and $\AFF/\SP K$,
    i.e. the pullback of affine schemes is affine.
  \end{itemize}
\end{prop}
\begin{proof}
  \textit{(Base Change)}
  The first adjunction is categorical. 
  For the restriction to affine schemes over $K$ and $L$,
  note that for any $K$-algebra $A$,
  \[
    \SP L \times_{\SP K} \SP A \text{ ``='' } \SP \brkt{L\otimes_K A}
  \]
\end{proof}

\end{document}