\documentclass[../main.tex]{subfiles}
\begin{document}
\begin{prop}[Big Zariski Site on $\M\SET$]
  
  For $X \in \M\SET$ and $\UU \subs \M\SET/X$ 
  a collection of morphisms into $X$,
  define $\UU \in \cov_\zar(X)$ when 
  ``$\UU$ is isomorphic to an open cover'',
  meaning there exists $\set{U_i}_{i\in\UU} \subs \open X$ such that 
  $\set{U_i}_{i\in\UU}$ is a cover of $X$ and 
  for all $i \in \UU$, $\brkt{i : s(i) \to X} \iso \brkt{U_i \to X}$
  in $\M\SET/X$.
  Then the above defines a Grothendieck pretopology of $\M\SET$.
  Specifically : 
  \begin{itemize}
    \item (Isomorphisms are Covers)
    For $X \in \M\SET$ and $\ph \in \M\SET(U,X)$,
    $\ph$ iso implies $\set{\ph} \in \cov_\zar(X)$.
    \item (Pullback of Covers)
    For all $\ph \in \M\SET(Y,X)$ and $\UU \in \cov_\zar(X)$,\newline
    $\ph\inv\UU := \set{Y\times_X s(i) \to Y}_{i \in \UU} 
    \in \cov_\zar(Y)$.
    \item (Composite of Covers)
    Let $X \in \M\SET$, $\UU \in \cov_\zar(X)$ and for each $i \in \UU$,
    let $\UU_i \in \cov_\zar(s(i))$.
    Then $\set{s(j_i) \to s(i) \to X \st i \in \UU, j_i \in \UU_i}
    \in \cov_\zar(X)$.
  \end{itemize}
  We will use $\M\SET_\zar$ to denote the site $\M\SET$ endowed with 
  the topology generated by the above pretopology. 
  We will call $\M\SET_\zar$ the \emph{big Zariski site}.
  $\XX \in \cov_\zar(X)$ are called \emph{Zariski covers of $X$}.
\end{prop}
\begin{proof}
  Only slightly non-trivial part is pullback of covers.
  Use opens and covers preserved under base change.
\end{proof}

\begin{rmk}[Intuition of Sheaves on $\M\SET_\zar$]
  For $X \in \M\SET$,
  if $X$ is to be a ``space'' then 
  for any other $Y \in \M\SET$ and open cover $\UU$ of $Y$,
  the data of a morphism $Y \to X$ should be the same as 
  a collection of morphisms $(U \to X)_{U \in \UU}$ that
  agree on pairwise intersection. 
  This is precisely what it means for 
  $\M\SET(-,X)$ to be a sheaf on the site $\M\SET_\zar$.
\end{rmk}

\begin{rmk}
  The following is a smaller site $\AFF_\zar$ on affine schemes,
  where open covers consists only of basic opens. 
  Since basic opens generate opens for affine schemes and 
  affine schemes generate $\M\SET$ with compatible notion of opens,
  sheaves on $\M\SET_\zar$ will be the same as sheaves of $\AFF_\zar$.
  This gives an easier check for when $X \in \M\SET$ is a 
  sheaf on $\M\SET_\zar$.
\end{rmk}

\begin{prop}[Small Zariski Site on $\AFF$]
  
  For $X \in \AFF$ and 
  $\UU \subs \AFF/X$,
  $\UU \in \cov_\zar(X)$ when 
  ``$\UU$ is isomorphic to a cover of $X$ by basic opens'',
  meaning there exists a cover $\set{X_{f_\io}}_{\io \in \UU}$
  where for all $\io \in \UU$,
  $\brkt{s(\io) \to X} \iso \brkt{D(f_\io) \to X}$ in $\AFF/X$.
  Then the above defines a Grothendieck pretopology on $\AFF$,
  specifically : 
  \begin{itemize}
    \item (Isomorphisms are Covers) 
    For all $X \in \AFF$ and $\io \in \AFF(U,X)$,
    $\io$ isomorphism implies $\set{\io} \in \cov_\zar(X)$.
    \item (Pullback of Covers)
    For all $\ph \in \AFF(Y,X)$ and $\UU \in \cov_\zar(X)$,\newline
    $\ph\inv\UU := \set{Y\times_X s(\io) \to Y \st \io \in \UU} 
    \in \cov_\zar(Y)$.
    \item (Composite of Covers)
    Let $\UU \in \cov_\zar(X)$ and for each $i \in \UU$,
    let $\UU_i \in \cov_\zar(s(i))$.
    Then $\set{s(j_i) \to s(i) \to X \st i \in \UU, j_i \in \UU_i}
    \in \cov_\zar(X)$.
  \end{itemize}
  We will use $\AFF_\zar$ to denote the site $\AFF$ with 
  the topology given by the above pretopology.
  We will call $\AFF_\zar$ the \emph{small Zariski site}.
  $\XX \in \cov_\zar(X)$ will be called \emph{basic Zariski covers of $X$}.
  \footnote{
    This is non-standard terminology, but helps avoid 
    confusion between the topology on $\AFF$ just defined and 
    the induced topology from $\M\SET_\zar$.
  }
\end{prop}
\begin{proof}
  UP of tensor products and localization.
\end{proof}

\begin{prop}[Sheaves on Big and Small Zariski Site are the Same]
  
  Let $X \in \M\SET$.
  Then $\M\SET(-,X) \in \SH(\M\SET_\zar)$ if and only if 
  $\M\SET(-,X) \in \SH(\AFF_\zar)$.
\end{prop}
\begin{proof}
  Forward implication follows since 
  the covers in $\AFF_\zar$ are covers in $\M\SET_\zar$.

  Now let $\M\SET(-,X) \in \SH(\AFF_\zar)$.
  Let $U \in \M\SET$ and $\UU \in \cov_\zar(U)$.
  Then for $(A,\al) \in \SP \darrow U$, 
  the pullback $\al\inv\UU$ of $\UU$ is a cover of $\SP A$ in the 
  big Zariski site. 
  The chain of isomorphisms to be justified is : 
  \begin{align*}
    \M\SET(U,X) 
    &\overset{(1)}{\iso} \LIM_{(A,\al) \in \SP\darrow U} \M\SET(\SP A,X)
    \overset{(2)}{\iso} \LIM_{(A,\al) \in \SP\darrow U} \LIM_{V,W \in \UU} 
      \M\SET(\al\inv V \cap \al\inv W, X) \\
    &\overset{(3)}{\iso} \LIM_{V,W \in \UU} \LIM_{(A,\al) \in \SP\darrow U} 
    \M\SET(\al\inv V \cap \al\inv W, X)
    \overset{(4)}{\iso} \LIM_{V,W \in \UU} \M\SET(V\cap W, X)
  \end{align*}
  
  $(1)$ Density of representables. $(3)$ Limits commute with limits.

  $(4)$ We know $\al\inv (V \cap W) = \al\inv V \cap \al\inv W$,
  so it suffices to prove the following. 
  \begin{lem}
    For $U \in \M\SET$ and $Z \in \SUB\M\SET(U)$, we have $
      Z = \LIM_{(A,\al) \in \SP\darrow U} \al\inv Z
    $
  \end{lem}
  \begin{proof1}
    The forgetful functor $\SP\darrow Z \to \SP\darrow U$
    is a ``section'' of the pullback functor $\SP\darrow U \to \SP\darrow Z$,
    meaning for $(A,\al) \in \SP\darrow Z$, 
    the following is a pullback diagram : 
    \begin{cd}
      Z \ar[r] & U \\
      \SP A \ar[u,"\al"] \ar[r,"\id{}"] & \SP A \ar[u]
    \end{cd}
    This implies pulling the diagram $\SP\darrow U$ back to $\SP\darrow Z$
    only introduces duplicate objects with identity morphisms in between them.
    Hence $\LIM_{(A,\al) \in \SP\darrow U} \al\inv Z = 
    \LIM_{(A_1,\al_1) \in \SP\darrow Z} \SP A = Z$ by 
    the density of representables. 
  \end{proof1}

  $(2)$ We need to show that $\M\SET(-,X)$ is a sheaf for 
  $\AFF$ with covers from the big Zariski site $\M\SET_\zar$.
  The key is that basic opens cover opens for affine schemes.

  Let $A \in \M\op$ and $\UU$ be a $\M\SET_\zar$-cover of $\SP A$.
  For each $i \in \UU$, let $I_i \in \ideal A$ with $i = D(I_i)$.
  Let $I := \bigsqcup_{i \in \UU} I_i$.
  Then since $\set{D(f)}_{f \in I_i}$ is a $\M\SET_\zar$-cover of 
  $i$ for every $i \in \UU$,
  $\set{D(f)}_{f \in I}$ is also a $\M\SET_\zar$-cover of $\SP A$.
  We then have the commutative diagram : 
  \begin{cd}
    \M\SET(\SP A,X) \ar[r] \ar[d,"\id{}"] & 
    \LIM_{i,j \in \UU} \M\SET(i \cap j, X) \ar[d,"\sim"] \\
    \M\SET(\SP A,X) \ar[r,"\sim"] & 
    \LIM_{f, g \in I} \M\SET(D(f) \cap D(g), X)
  \end{cd}
  where the horizontal isomorphism to due to 
  $\M\SET(-,X)$ being a sheaf on $\AFF_\zar$.

  It remains to justify the vertical isomorphism.
  To do this, we apply the same argument as we're trying to do now,
  but on $i \cap j$.
  It's easy to see that $\set{D(f) \cap D(g)}_{f\in I_i,g\in I_j}$
  covers $i \cap j$,
  so we get  
  \begin{align*}
    \M\SET(i\cap j,X) 
    &\overset{}{\iso} 
      \LIM_{(A_1,\al_1) \in \SP\darrow (i \cap j)} \M\SET(\SP A_1,X)
    \overset{}{\iso} 
      \LIM_{(A_1,\al_1) \in \SP\darrow (i \cap j)} 
      \LIM_{f \in I_i,g\in I_j} 
      \M\SET(D(\al_1^\flat(f)) \cap D(\al_1^\flat(g)), X)\\
    &\overset{}{\iso} 
      \LIM_{f \in I_i, g\in I_j} \LIM_{(A_1,\al_1) \in \SP\darrow U} 
      \M\SET(\al\inv_1 \brkt{D(f) \cap D(g)}, X) 
    \overset{(4)}{\iso} \LIM_{f \in I_i, g\in I_j} \M\SET(D(f)\cap D(g), X)
  \end{align*}
  where $(4)$ is as before.
  
\end{proof}
\end{document}